% Created 2015-02-13 Fri 11:15

\documentclass[nofonts]{tufte-handout}
\usepackage{fontspec}
\usepackage{ifxetex}
\setmainfont{Minion Pro}
\usepackage{lipsum}

\ifxetex
  \newcommand{\textls}[2][5]{%
    \begingroup\addfontfeatures{LetterSpace=#1}#2\endgroup
  }
  \renewcommand{\allcapsspacing}[1]{\textls[15]{#1}}
  \renewcommand{\smallcapsspacing}[1]{\textls[10]{#1}}
  \renewcommand{\allcaps}[1]{\textls[15]{\MakeTextUppercase{#1}}}
  \renewcommand{\smallcaps}[1]{\smallcapsspacing{\scshape\MakeTextLowercase{#1}}}
  \renewcommand{\textsc}[1]{\smallcapsspacing{\textsmallcaps{#1}}}
\fi


\usepackage{polyglossia}
\setmainlanguage{english}
\PolyglossiaSetup{greek, english}{indentfirst=false}

% Some extras:
\usepackage{hyperref} % in-document links (for TOC etc)
\usepackage{lipsum} % lorem lipsum ready-text
\usepackage{ulem} % underline emphasis

% HEADER ENDS HERE ===============================================
% ================================================================
\author{IOANNIS ZANNOS}
\date{\today}
\title{L’histoire des illustrations}
\hypersetup{
  pdfkeywords={},
  pdfsubject={},
  pdfcreator={Emacs 24.4.1 (Org mode 8.2.10)}}
\begin{document}

\maketitle


Les illustrations ont longtemps été considérées comme un élément purement accessoire dans NIA ; ainsi, Jean Ferry, dans l’édition Pauvert de 1963, leur avait conféré, en les regroupant à la fin du livre, un caractère tout à fait indépendant du texte, n’hésitant pas à parler, dans une note explicative, « d’illustrations fantômes », puisque destinées à rester entre les pages non coupées du volume\footnote{Cf. Raymond Roussel, Nouvelles impressions d’Afrique, op. cit., p. 89.}. Roussel aurait pris les décisions concernant le brochage du volume, l’ajout des illustrations et la réédition de « Mon âme » au dernier moment, confronté à l’évidente minceur d’un texte de seulement soixante-dix pages \footnote{Sur cette question, on se reportera à la biographie de François Caradec, Raymond Roussel, 1997, p. 365-sq.}.

Il faudra attendre d’avoir analysé minutieusement les manuscrits de NIA, récemment mis à la disposition du public par la Bibliothèque nationale de France\footnote{Les quatorze années qui se sont écoulées depuis la découverte des manuscrits de NIA en 1991, jusqu’à leur mise en ordre par la BnF en 2005, donnent aussi une idée des difficultés que pose cet ouvrage.}, pour établir la genèse du volume et, éventuellement, le moment exact de l’apparition des images ; pour l’heure, dans les premiers parcours que j’ai pu faire de ces documents, je n’ai réussi à repérer aucune référence aux illustrations, et les risques qu’ils n’en comportent aucune sont élevés.


Ces manuscrits montrent que, pour une grande partie, la rédaction du dernier ouvrage était déjà très avancée vers 1918, et qu’en 1927 Eugène Vallée, le prote de Lemerre qui s’était occupé de la plupart des livres de Roussel, travaillait déjà à la composition du volume dont la première édition porte le 30 juin 1932 comme date d’achevé d’imprimer\footnote{Roussel fixe dans son ouvrage posthume Comment j’ai écrit certains de mes livres (Paris, Pauvert, 1963 [Paris, Lemerre, 1935]) les dates d’écriture de NIA entre 1915 et 1928. Mais les documents du Fonds Roussel de la BnF permettent, aujourd’hui, de savoir que la composition matérielle du volume débute déjà en 1927 et elle est pratiquement terminée en 1931. Du travail réalisé par l’auteur pendant ces quinze années, la BnF conserve cent soixante feuillets autographes et dactylographiés. Cf. Annie Angremy, « La malle Roussel du bric à brac au décryptage », Revue de la BN. Découvrir Raymond Roussel, nº 43 (printemps 1992), p. 37-49 ; Laurent Busine, Raymond Roussel. Contemplator Enim, 1995, p. 21.}. La mise au point matérielle du volume se prolonge donc plus de quatre années, pendant lesquelles l’auteur aurait largement eu le temps d’en méditer la forme finale, pour le moins recherchée. Il est difficile d’imaginer Roussel s’adonnant, pour une telle tâche, aux plaisirs de l’improvisation.


La genèse de la commande des cinquante-neuf images qui accompagnent le texte à travers l’agence de détectives Goron au dessinateur Henri-Achille Zo\footnote{François Caradec, Raymond Roussel, op. cit., p. 365-sq. Le biographe rappelle que Henri-A. Zo avait illustré le roman de Pierre Loti, Ramuntcho, et l’admiration que Roussel portait à cet auteur pourrait avoir compté à l’heure de choisir justement ce dessinateur.} que Roussel, lui, ne rencontrera jamais, aurait donc été longuement mûrie. Laurent Busine en a donné une interprétation assez plausible :

\begin{quote}
[\ldots{}] je pense que Raymond Roussel préméditait scrupuleusement cette commande, mais qu’il n’a pu en établir avec précision les termes qu’une fois les quatre chants des Nouvelles impressions d’Afrique composés. Il sut à ce moment combien de pages de texte le livre comporterait et donc combien de dessins il devait commander ; Raymond Roussel put alors déterminer, grâce à la pagination, quelles illustrations il désirait voir exécuter, quels sujets précisément il choisirait\footnote{Laurent Busine, Raymond Roussel. Contemplator Enim,op. cit., p. 17.}.
\end{quote}

L’idée d’une lente préméditation de la commande posée par Laurent Busine relève de l’hypothèse vraisemblable, mais pour ne s’en tenir qu’aux faits il convient d’insister, comme il n’hésite pas à le faire, d’ailleurs, sur l’expérience éditoriale de Roussel qui avait publié tous ses livres précédents chez le même éditeur à compte d’auteur (des éditions qui contribuèrent sans doute à sa ruine financière, les Nouvelles impressions étant en ce sens comme les premières des « impressions à fric ») et dont on sait qu’il se montrait extrêmement attentif au moindre détail concernant la publication de chacune de ses oeuvres. Quand il s’apprête à publier NIA, Roussel est donc un auteur qui a une bonne connaissance des processus de composition d’un volume et qui ne saurait se montrer indifférent aux questions touchant au brochage, au pliage d’un livre qui lui avait coûté tant d’efforts. « Il fallait [par exemple] que Raymond Roussel connaisse avec exactitude la présentation des pages imprimées, la disposition du texte et des notes… pour décider des images à faire réaliser\footnote{Id.} », de manière à ce que chaque illustration corresponde à des vers sélectionnés dans la page de texte qui la précédait.
% Emacs 24.4.1 (Org mode 8.2.10)
\end{document}
